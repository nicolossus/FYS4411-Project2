%================================================================
\section{Introduction}\label{sec:Introduction}
%================================================================

- Short motivation

- In this project ... 

- Outline structure


In recent years, using machine learning to solve quantum mechanical systems has become of great interest for many. The complexity of the many-body quantum wave function increases exponentially with the number of particles, and is therefore computationally very costly, often intractable for practical purposes. Machine learning models may be of practical use in this, as they are designed to find statistical correlations between high-dimensional feature space. Recently, representing wave functions as a Restricted Boltzmann Machine (RBM) has been presented by G. Carleo and M. Troyer where they applied it to quantum mechanical spin lattice system of the Ising model and Heisenberg model \citep{Carleo_2017}. They dubbed the approach of using a neural network to represent a quantum state \textit{Neural network Quantum States} (NQS). In this project, we will fit an RBM to two systems; one electron in a one-dimensional harmonic oscillator trap and two interacting electrons confined in an isotropic two-dimensional harmonic oscillator trap. We will be using a reinforcment learning approach using a generative method to evaluate and update the RBM, and utilize the variational principle to find the system's ground state. Specifically, the RBM will act as an approximation to the wave function which we will perform calculations on using different Markov Chain Monte Carlo (MCMC) approaches.  %The wave function of the system can be formulated as an artificial neural network, which is called the Neural-Network Quantum State (NQS), represented by the RBM. The RBM will act as an 
This project can be thought of as an extension of \citep{project1}, and for reviews and descriptions of principles regarding the MCMC approaches we will refer to the previous project. 
We will next (in \autoref{sec:Theory}) have a look at the systems we are performing calculations on, then look at theoretical applicability of the RBM applied as an NQS. In \autoref{sec:Method} we will display short descriptions of the methods applied in this project, as many of them are very similar to those used in our previous project \citep{project1}. \autoref{sec:Results} will display the results provided by the NQS and discussions regarding those, where \autoref{sec:Conclusion} will provide conclusions to the approach. Finally, in \autoref{sec:Future} we will look at interesting avenues for future research. 

\iffalse
We will test our implementation against a Variational Monte Carlo method with optimization algorithms such as ADAM (which will also be used to find an upper limit for the ground state?)

As quantum mechanical systems become more complex, it becomes exponentially harder to solve the problem numerically, and only the simplest of models have an analytical solution. 

1. G. Carleo and M. Troyer, Science 355, Issue 6325, pp. 602-606 (2017)
\fi