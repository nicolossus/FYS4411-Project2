%================================================================
%------------------------- Abstract -----------------------------
%================================================================
\begin{abstract}
The many-body wave function increases exponentially in complexity with the number of particles, and therefore, clever approximations to it is of great interest. The universal approximation theorem states that any function can be approximated to an arbitrary error by a neural network. We will therefore seek to implement an approach based on a machine learning method. 
An approach based on a neural network for solving the quantum mechanical wave function is still a relatively new, but an increasingly interesting prospect \citep{Saito_2018}. 
This project analyzes two systems of electrons, a quantum dot in a one-dimensional harmonic oscillator trap and a pair of interacting electrons in an isotropic two-dimensional harmonic oscillator trap. We analyze the systems using an unsupervised learning method, the restricted Boltzmann machine (RBM), to simulate the wave function (known as a Neural-network Quantum State\citep{Carleo_2017}), and generate upper bound estimates to the ground state energies using two Markov Chain Monte Carlo algorithms. We perform some coarse searches in the space of hyper parameters, like learning rate, batch-sizes for optimization and number of neurons in hidden layer. After finding the optimal set of parameters, we find that we approximate the ground state energy for the quantum dot in a single dimension to a high degree of precision. We find the best approximation to the ground state energy to be $E_0 = 0.499999 \pm 3\cdot10^{-6}$ a.u, using the RWM sampling algorithm. For the system of two interacting electrons in two-dimensional space we find the best approximation, again via the RWM sampling algorithm, to be $E_0 = 3.059 \pm 0.008$ a.u. Knowing the true ground state energy to be $E_0=3.0$ a.u \citep{PhysRevA.48.3561}, there is still plenty of room for improvement.
\end{abstract}